\documentclass[a4paper,12pt]{article}

%\usepackage[utf8]{inputenc}
%\usepackage{lipsum}
\usepackage{amsmath,amsthm}
\usepackage[english]{babel}

%\usepackage[T1]{fontenc}
\usepackage{graphicx}
\usepackage{amssymb}

\newcommand\norm[1]{\left\lVert#1\right\rVert}
\usepackage{calc}
\usepackage{rotating}
\usepackage[usenames,dvipsnames]{color}
\usepackage{fancyhdr}
%\usepackage{subfigure}
\usepackage{hyperref}
\usepackage{longtable}
\usepackage{svg}
\usepackage{float}
\usepackage{calc}
\usepackage{rotating}
\usepackage[usenames,dvipsnames]{color}
\usepackage{fancyhdr}
%\usepackage{subfigure} 
\usepackage{hyperref}

\usepackage[backref=false,
			style=numeric-comp,
            sorting=none]{biblatex}
\bibliography{journals,phd-references}

\usepackage{xcolor}
\hypersetup{
    colorlinks,
    linkcolor={red!50!black},
    citecolor={blue!50!black},
    urlcolor={blue!80!black}
}
\graphicspath{{images/}}


\begin{document}
 
\begin{align}
\label{first}
v_{x}&=\frac{(v_{l}+v_{r})}{2}\\
\label{second}
v_{l}&=v_{x}+\omega.d\\
\label{third}
v_{r}&=v_{x}-\omega.d\\
\intertext{using equations \ref{second} and \ref{third}, we get}
\label{fourth}
v_{l}-v_{r}&=2\omega.d\\  
\intertext{and we also know that}
\frac{v_{r}}{x}&=\frac{v_{l}}{2d+x}\\
\intertext{which implies,} 
x&=2d\frac{v_{r}}{(v_{l}-v{r})}\\
d+x&=d\frac{(v_{l}+v_{r})}{(v_{l}-v_{r})}\\
\intertext{replacing $d+x$ by $r$ and using \ref{first} and \ref{fourth}}
r&=\frac{v_{x}}{\omega}
\end{align}
\end{document}
